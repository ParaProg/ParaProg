\documentclass[]{article}
\usepackage[english,ngerman]{babel}
\usepackage[utf8]{inputenc}
\usepackage{graphicx}
\linespread{1.2}

%opening
\title{Bierfass-Simulation}
\author{Mark Geiger \and Rico Fritzsche}

\begin{document}

\maketitle

\begin{abstract}
Im Kurs Parallele Programmierung soll der Geschwindigkeitsunterschied verschiedener Umsetzungen der Matrixmultiplikation demonstriert werden. Dazu wurde ein Testprogramm erstellt, welches unterschiedliche Konzepte umsetzt und zeitlich bewertet. Zur Auswertung wurde ein Python-Skript erstellt, welches die Zeit als Funktion der Matrixgröße darstellt.
\end{abstract}

\section*{Umsetzung}
 



\section*{Ausführung}
\textit{make}\\
\textit{make run}\\
\textit{make plot}

\begin{figure}[h]
	\centering
	\includegraphics[width=1\linewidth]{}
	\caption{Ergebnis der Testimplementation, dargestellt mittels Python-Scripts}
	\label{pic:ergebnis}
\end{figure}
\end{document}
